%%%%%%%%%%%%%%%%%%%%%%%%%%%%%%%%%%%%%%%%%%%%%%%%%%%%%%%%%%%%%%%%%%%%%%
% Modelo de ARTIGO GERAL da newsletter da Comissão
% de Pedometria da Sociedade Brasileira de Ciência do Solo
%
% Esse modelo deve ser utilizado para a construção de ARTIGOS GERAIS.
% 
% Recomendações gerais:
% A edição desse documento deve ser feita utilizando um editor LaTeX
% como o RStudio.
% Linhas precedidas por '%' não precisam ser deletados pois não
% são reconhecidos durante a copilação do documento.
% Elementos como '\title{}' constituem comandos e não devem
% ser alterados, exceto o conteúdo entre parênteses.
% Ex.: '\title{Título do seu artigo}' = '\title{O que é Pedometria?}'
%
% Coloque seu nome como nome do arquivo. Ex. 'artigo-alessandro.tex'.
% O arquivo final deve ser enviado compactado junto das figuras.
%
% Language: Latex
%%%%%%%%%%%%%%%%%%%%%%%%%%%%%%%%%%%%%%%%%%%%%%%%%%%%%%%%%%%%%%%%%%%%%%

% Cabeçalho do seu artigo. Substitua apenas as informações entre parênteses.
\title{Título do seu artigo}
\subtitle{Subtítulo do seu artigo}
\author{por primeiro autor e segundo autor}
\maketitle



% Aqui vai a sua foto. Não adicione o caminho, apenas o nome do
% arquivo e o formato.
\begin{wrapfigure}{l}{0.15\textwidth}
\includegraphics[width=0.15\textwidth]{sua-foto-aqui.jpg}
\end{wrapfigure}



% Aqui começa seu texto.
Era uma vez...



% Se você tiver seções use os comandos a seguir. Do contrário,
% simplesmente apague-os.
\subsection{Título da sua subseção}
\label{subsec:rótulo-da-sua-subseção}



% Aqui você continua seu texto.
Era uma vez...



% Com os comandos abaixo você insere uma figura.
\begin{figure}[htbp]
   \centering
   % aqui vocÊ identifica o aquivo no formato PNG ou PDF:
   \includegraphics[scale=0.8]{sua-figura-aqui.jpg}

   % aqui você dá um título à figura:
   \caption{Esse é o título da miha figura.}

   % e aqui você define o rótulo da figura, que é usado para criar
   % links no documento com a função '\ref{}':
   \label{fig:rótulo-da-figura}
\end{figure}



% Aqui você continua seu texto.
E para citar a figura use o comando \ref{fig:rótulo-da-figura}.



% Se você tiver mais uma seção, use os comandos a seguir. Do
% contrário, simplesmente apague-os.
\section{Nome da minha outra seção}
\label{sec:rótulo-da-minha-outra-seção}



% Aqui você continua seu texto.
Era uma vez...



% Para adicionar linhas de comando de algum software, use a 
% função 'smallverbatim':
\begin{smallverbatim}
# cole aqui seu comando:
GRASS 5.7.-cvs:~ > d.mon x0
using default visual which is TrueColor
ncolors: 65536
Graphics driver [x0] started
GRASS 5.7.-cvs:~ > d.rast wcm
 100%
GRASS 5.7.-cvs:~ >

etc. 
\end{smallverbatim}



% Aqui você continua seu texto.
Era uma vez...



% Se você quiser citar alguma referência bibliográfica use a função
% '\cite{nome:ano}' e '\citep{nome:ano}'. Por exemplo:
Os resultados de \cite{highlander:2003} são contraditórios.



% Se a citação deve aparecer entre parêntese, então use o seguinte:
Os resultados são contraditórios \citep{highlander:2003}.



% Aqui termina o seu texto e começa a sua lista de referências
% bibliográficas.
\begin{footnotesize}
\begin{thebibliography}{99}



% primeiro item da sua lista:
% a primeira linha descreve como a citação aparece no texto
\bibitem[Highlander et~al. (2003) Highlander, Batman, Wolverine, Hulk]{highlander:2003}
% essas três linhas definem como a citação aparece nas referências
L.M. Highlander, M.G. Batman, A.S. Wolverine, S.P. Hulk (2003)
\newblock A pedometria é muito legal.
\newblock {\em Revista Pedometria} 24: 69-96.



% copie e edite as linhas acima para adicionar mais referências


\end{thebibliography}
\end{footnotesize}
% aqui termina sua lista de referências!



% aqui vão as suas informações
\address{Seu Nome\\
  Sua Organização\\
  \url{seu website}\\
  \email{seu endereço de e-mail}}



%%% Local Variables: 
%%% mode: latex
%%% TeX-master: documento-principal.tex
%%% End: 

