%%%%%%%%%%%%%%%%%%%%%%%%%%%%%%%%%%%%%%%%%%%%%%%%%%%%%%%%%%%%%%%%%%%%%%
% Modelo de ARTIGO DESCRITIVO DE NOVAS PUBLICAÇÕES da
% newsletter da Comissão de Pedometria da Sociedade Brasileira de
% Ciência do Solo
%
% Esse modelo deve ser utilizado para a construção de ARTIGO
% DESCRITIVO DE NOVAS PUBLICAÇÕES.
% 
% Recomendações gerais:
% A edição desse documento deve ser feita utilizando um editor LaTeX
% específico ou o RStudio.
% Comentários precedidos por '%' não precisam ser deletados pois não
% são reconhecidos durante a copilação do documento.
% Elementos como '\title{}' constituem comandos e não devem
% ser alterados, exceto o conteúdo entre parênteses.
% Ex.: '\title{Título do seu artigo}' = '\title{O que é Pedometria?}'
% 
% Language: Latex
% 
%%%%%%%%%%%%%%%%%%%%%%%%%%%%%%%%%%%%%%%%%%%%%%%%%%%%%%%%%%%%%%%%%%%%%%

\title{Novas publicações em pedometria}
\maketitle

% Se você tiver algum texto de apresentação, este é o espaço para inserí-lo!

% Primeira publicação
\section{O título da publicação}

\subsection{Citação}
G. Xadrez (2013) A cor do Gato Xadrez. {\em Revista Científica do Xadrez} 04:01-09. 

\subsection{Resumo}
Era uma vez um gato xadrez. Quer que eu conte outra vez?

% Repita a estrutura acima para inserir novos títulos.

% Com os comandos abaixo você insere uma figura.
\begin{figure}[htbp]
   \centering
   \includegraphics[scale=0.8]{sua-figura-aqui.jpg} % aqui você identifica o aquivo no formato PNG ou PDF
   \caption{Esse é o título da miha figura.} % aqui você dá um título à figura
   \label{fig:rótulo-da-figura} % aqui você define o rótulo da figura, que é usado para criar links no documento com a função '\ref{}'
\end{figure}

% E para citar a figura use o comando \ref{fig:rótulo-da-figura}.

% Se você quiser citar alguma referência bibliográfica use a função
% '\cite{nome:ano}' e '\citep{nome:ano}'. Por exemplo:
Os resultados de \cite{highlander:2003} são contraditórios.



% Se a citação deve aparecer entre parêntese, então use o seguinte:
Os resultados são contraditórios \citep{highlander:2003}.




% Bibliografia
\begin{footnotesize}
\begin{thebibliography}{99}



% Lista das publicações descritas em ordem alfabética:
 
\bibitem[Herborg et~al. (2003) Herborg, Bentley, Clare, Rushton]{herborg:2003} 
L.M. Herborg, M.G. Bentley, A.S. Clare, S.P. Rushton (2003)
\newblock The spread of the Chinese mitten crab (Eriocheir sinensis) in
Europe; the predictive value of an historical data set.
\newblock {\em Hydrobiologia} 503: 21-28.

\bibitem[Lurz et~al. (2001) Lurz, Rushton, Wauters, Bertolino, Currado,
Mazzoglio, Shirley]{lurz:2001} P.W.W. Lurz, S.P. Rushton, L.A. Wauters, I. Currado,
P. Mazzoglio, M.D.F. Shirley (2001)
\newblock Predicting grey squirrel expansion in North Italy: a spatially
explicit modelling approach.
\newblock {\em Landscape Ecology} 16: 407-420.



\end{thebibliography}
\end{footnotesize}



%%% Local Variables: 
%%% mode: latex
%%% TeX-master: documento-principal.tex
%%% End: 
