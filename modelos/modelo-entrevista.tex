%%%%%%%%%%%%%%%%%%%%%%%%%%%%%%%%%%%%%%%%%%%%%%%%%%%%%%%%%%%%%%%%%%%%%%
% Modelo de ENTREVISTA da newsletter da Comissão
% de Pedometria da Sociedade Brasileira de Ciência do Solo
%
% Esse modelo deve ser utilizado para a construção de ENTREVISTAS.
% 
% Recomendações gerais:
% A edição desse documento deve ser feita utilizando um editor LaTeX.
% Comentários precedidos por '%' não precisam ser deletados pois não
% são reconhecidos durante a copilação do documento.
% Elementos como '\subtitle{}' constituem comandos e não devem
% ser alterados, exceto o conteúdo entre parênteses.
% Ex.: '\subtitle{com Nome do Entrevistado}' = '\subtitle{com João Maria}'
%
% Coloque o nome do entrevistado como nome do arquivo.
% Ex. 'entrevista-joao-maria.tex'.
% O arquivo final deve ser enviado compactado junto das figuras.
%
% Use o comando '\pergunta{}' para as perguntas.
%
% Use o comando '\resposta{Nome}{Resposta}' para as respostas.
%
% Language: Latex 
%%%%%%%%%%%%%%%%%%%%%%%%%%%%%%%%%%%%%%%%%%%%%%%%%%%%%%%%%%%%%%%%%%%%%%

% Entrevista
\title{Entrevista}
\subtitle{com Nome do entrevistado}
\maketitle

% Coloque uma foto do entrevistado aqui (modelo Lattes)!
\begin{wrapfigure}{l}{0.15\textwidth}
\includegraphics[width=0.15\textwidth]{figuras/foto-do-entrevistado.png}
\end{wrapfigure}

% Aqui começa seu texto de apresentação do entrevistado.
Era uma vez um gato xadrez.

% Primeira pergunta
\pergunta{Quer que eu conte outra vez?}

% Aqui você coloca a resposta da primeira pergunta
\resposta{Nome do Entrevistado}{Não, não quero que conte outra vez!}

% Com os comandos abaixo você insere uma figura.
\vspace{5mm}
\begin{figure}[htbp]
   \centering
   % aqui vocÊ identifica o aquivo no formato PNG ou PDF:
   \includegraphics[scale=0.8]{sua-figura-aqui.jpg}

   % aqui você dá um título à figura:
   \caption{Esse é o título da miha figura.}

   % e aqui você define o rótulo da figura, que é usado para criar
   % links no documento com a função '\ref{}':
   \label{fig:rótulo-da-figura}
\end{figure}



% Aqui você continua com a próxima pergunta
\pergunta{Mas o gato xadrez da figura \ref{fig:rótulo-da-figura} é o seu gato xadrez?}

% Aqui você coloca a resposta
\resposta{Nome do Entrevistado}{Não, aquele é um gato amarelo!}

% Se você quiser citar alguma referência bibliográfica use a função
% '\cite{nome:ano}' e '\citep{nome:ano}'. Por exemplo:
Os resultados de \cite{highlander:2003} são contraditórios.

% Se a citação deve aparecer entre parêntese, então use o seguinte:
Os resultados são contraditórios \citep{highlander:2003}.


% Aqui termina o seu texto e começa a sua lista de referências
% bibliográficas.
\begin{footnotesize}
\begin{thebibliography}{99}



% primeiro item da sua lista:
% a primeira linha descreve como a citação aparece no texto
\bibitem[Highlander et~al. (2003) Highlander, Batman, Wolverine, Hulk]{highlander:2003}
% essas três linhas definem como a citação aparece nas referências
L.M. Highlander, M.G. Batman, A.S. Wolverine, S.P. Hulk (2003)
\newblock A pedometria é muito legal.
\newblock {\em Revista Pedometria} 24: 69-96.



% copie e edite as linhas acima para adicionar mais referências


\end{thebibliography}
\end{footnotesize}
% aqui termina sua lista de referências!



%%% Local Variables: 
%%% mode: latex
%%% TeX-master: documento-principal.tex
%%% End: 