\title{Pedometria na SBCS}
\maketitle
\newcommand{\estrutura}{\href{http://www.sbcs.org.br/a-sbcs/estatuto/}{estrutura científica}}
\newcommand{\SBCS}{\href{http://www.sbcs.org.br/}{SBCS}}
\newcommand{\IUSS}{\href{http://www.iuss.org/}{IUSS}}
\newcommand{\boletim}{\href{http://www.iuss.org/images/stories/IUSS\%20Bulletin\%201\%20-\%20117/00000097.pdf}{Boletim 97}}
\newcommand{\EspacoTempo}{\href{http://www.sbcs.org.br/comissoes-especializadas/divisao-1-solo-no-espaco-e-no-tempo/}{Espaço e no Tempo}}
\newcommand{\PedometronOnze}{\href{http://www.pedometrics.org/Pedometron/pedometron11.pdf}{Pedometron 11}}
\newcommand{\PedometronDoze}{\href{http://www.pedometrics.org/Pedometron/pedometron12.pdf}{Pedometron 12}}

A atual \estrutura{} da Sociedade Brasileira de Ciência do Solo (\SBCS) foi definida na Assembleia Geral dos Congressos Brasileiros de Ciência do Solo realizados em Fortaleza (2009) e Uberlândia (2011). Na época, o objetivo principal foi adequar-se a estrutura científica da União Internacional de Ciência do Solo (\IUSS), a qual fora definida no Congresso Mundial de Ciência do Solo realizado na Tailândia em 2002. Tal estrutura inclui Divisões, Comissões, Grupos de Trabalho, e Comitês Permanentes (\boletim). São as Divisões, as Comissões e os Grupos de Trabalho os responsáveis por desenvolver as atividades científicas da SBCS e da IUSS.

Hoje são quatro as Divisões da SBCS e da IUSS, cada uma subdividida em Comissões. A Divisão I da SBCS, chamada Solo no \EspacoTempo, possui três Comissões: Gênese e Morfologia do Solo, Levantamento e Classificação do Solo, e Pedometria.

A coordenação da Divisão I é a seguinte:

\begin{itemize}
 \item Diretor: Lúcia Helena Cunha dos Anjos - UFRRJ
 \item Vice-diretor: Humberto Gonçalves dos Santos - EMBRAPA
 \item Membro titular: Elpídio Inácio Fernandes Filho - UFV
 \item Membro suplente: Cristiane Valéria de Oliveira - UFMG
\end{itemize}

\subsection{Comissão de Pedometria}

A Comissão de Pedometria da IUSS foi criada no Congresso Mundial de Ciência do Solo realizado na Tailândia em 2002 a partir do Grupo de Trabalho em Pedometria, um dos mais ativos na época. Muitas foram as discussões sobre a Divisão a qual a Comissão de Pedometria deveria pertencer (\PedometronOnze{} e \PedometronDoze). Isso porque a pedometria constitui um campo que abrange muitos outros campos da ciência do solo, como estatística, matemática, física, química, informática, biologia, mineralogia, geografia \textit{et cetera}. Apesar da multitude de campos de atuação, a nova estrutura científica da IUSS exige que as Comissões façam parte de uma única Divisão. Como a maior parte das contribuições da pedometria estão relacionadas à variabilidade espaço-temporal das propriedades do solo, o consenso foi de que a Divisão I seria a mais apropriada.

A Comissão de Pedometria da SBCS possui em seu escopo os seguintes temas:

\begin{itemize}
 \item Fotointerpretação e sensoriamento remoto
 \item Mapeamento digital de solos
 \item Variabilidade espacial e temporal de atributos de solo (geoestatítica)
 \item Análise espectral (refletância)
 \item Processamento de dados de solos
\end{itemize}

A coordenação da Comissão de Pedometria é a seguinte:

\begin{itemize}
 \item Coordenador: Maria de Lourdes Mendonça Santos - EMBRAPA
 \item Vice-coordenador: Elpídio Inácio Fernandes Filho - UFV
 \item Membros titulares: César da Silva Chagas - EMBRAPA, e José Alexandre Melo Demattê - ESALQ/USP
 \item Membros suplentes: Gustavo de Mattos Vasques - EMBRAPA, e Ricardo Simão Diniz Dalmolin - UFSM
\end{itemize}

\subsection{Grupo de Trabalho em Mapeamento Digital de Solos}

Os Grupos de Trabalho são grupos científicos criados em caráter temporário para tratar de problemas específicos da ciência do solo. Atualmente a IUSS possui 17 Grupos de Trabalho atuando nos mais diversos temas.

O primeiro Grupo de Trabalho da SBCS foi criado durante o Congresso Brasileiro de Ciência do Solo realizado em Florianópolis (2013). Trata-se do Grupo de Trabalho em Mapeamento Digital de Solos (GT-MDS). A solicitação foi feita pela Rede Brasileira de Pesquisa em Mapeamento Digital de Solos (RedeMDS), constituída por pesquisadores de diversas instituições de ensino e pesquisa brasileiras.

O GT-MDS possui como missão elaborar projetos de ação conjunta e integrada para o mapeamento digital dos solos do Brasil e gerar sinergia entre os pesquisadores brasileiros para promover o avanço da pesquisa em mapeamento digital de solos.

Especificamente, o GT-MDS possui como objetivos:

\begin{itemize}
 \item Consolidar a Rede Brasileira de Mapeamento Digital de Solos do Brasil
 \item Realizar o mapeamento digital de atributos do solo em todo o território brasileiro
 \item Realizar o mapeamento digital de classes de solos
 \item Propor metodologia e validação de mapeamento digital de classes e/ou atributos de solos em áreas pilotos do território brasileiro
 \item Realizar treinamento em mapeamento digital de solos
\end{itemize}

A fim de cumprir sua missão e alcançar seus objetivos, o GT-MDS procura realizar encontros anuais. Estes encontros geralmente ocorrem concomitantes a outros eventos de abrangência nacional a fim de estimular a interdisciplinaridade e otimizar os recursos disponíveis. Nos anos em que é realizado o Congresso Brasileiro de Ciência do Solo, o GT-MDS reúne-se para avaliar as atividades científicas passadas, definir as atividades científicas futuras, e faz um relato das mesmas ao Coordenador Geral da Divisão I.

A coordenação atual do GT-MDS é composta pelos seguintes pesquisadores:

\begin{itemize}
 \item Coordenador geral: Márcio Francelino - UFV
 \item Coordenador Regional I: Ricardo Simão Diniz Dalmolin - UFSM
 \item Coordenador Regional II: José Alexandre M. Demattê - ESALQ/USP
 \item Coordenador Regional III: Elpidio Inácio Fernandes Filho - UFV
 \item Coordenador Regional IV: Gustavo de Mattos Vasques - EMBRAPA SOLOS
 \item Coordenador Regional V: Gustavo Souza Valladares - UFPI
\end{itemize}
%%% Local Variables: 
%%% mode: latex
%%% TeX-master: 5th-edition.tex
%%% End: